%%%%%%%%%%%%%%%%%%%%%%%%%%%%%%%%%%%%%%%%%%%%%%%%%%%%%%%%%%%%%%%%%%%%%%%
%%%%%%%%%%%%%%%%%%%%%% Simple LaTeX CV Template %%%%%%%%%%%%%%%%%%%%%%%%
%%%%%%%%%%%%%%%%%%%%%%%%%%%%%%%%%%%%%%%%%%%%%%%%%%%%%%%%%%%%%%%%%%%%%%%%

%%%%%%%%%%%%%%%%%%%%%%%%%%%%%%%%%%%%%%%%%%%%%%%%%%%%%%%%%%%%%%%%%%%%%%%%
%% NOTE: If you find that it says                                     %%
%%                                                                    %%
%%                           1 of ??                                  %%
%%                                                                    %%
%% at the bottom of your first page, this means that the AUX file     %%
%% was not available when you ran LaTeX on this source. Simply RERUN  %%
%% LaTeX to get the ``??'' replaced with the number of the last page  %%
%% of the document. The AUX file will be generated on the first run   %%
%% of LaTeX and used on the second run to fill in all of the          %%
%% references.                                                        %%
%%%%%%%%%%%%%%%%%%%%%%%%%%%%%%%%%%%%%%%%%%%%%%%%%%%%%%%%%%%%%%%%%%%%%%%%

%%%%%%%%%%%%%%%%%%%%%%%%%%%% Document Setup %%%%%%%%%%%%%%%%%%%%%%%%%%%%

% Don't like 10pt? Try 11pt or 12pt
\documentclass[10pt]{article}

% This is a helpful package that puts math inside length specifications
\usepackage{calc}


% Simpler bibsection for CV sections
% (thanks to natbib for inspiration)
\makeatletter
\newlength{\bibhang}
\setlength{\bibhang}{1em}
\newlength{\bibsep}
 {\@listi \global\bibsep\itemsep \global\advance\bibsep by\parsep}
\newenvironment{bibsection}%
        {\vspace{-\baselineskip}\begin{list}{}{%
       \setlength{\leftmargin}{\bibhang}%
       \setlength{\itemindent}{-\leftmargin}%
       \setlength{\itemsep}{\bibsep}%
       \setlength{\parsep}{\z@}%
        \setlength{\partopsep}{0pt}%
        \setlength{\topsep}{0pt}}}
        {\end{list}\vspace{-.6\baselineskip}}
\makeatother

% Layout: Puts the section titles on left side of page
\reversemarginpar

%
%         PAPER SIZE, PAGE NUMBER, AND DOCUMENT LAYOUT NOTES:
%
% The next \usepackage line changes the layout for CV style section
% headings as marginal notes. It also sets up the paper size as either
% letter or A4. By default, letter was used. If A4 paper is desired,
% comment out the letterpaper lines and uncomment the a4paper lines.
%
% As you can see, the margin widths and section title widths can be
% easily adjusted.
%
% ALSO: Notice that the includefoot option can be commented OUT in order
% to put the PAGE NUMBER *IN* the bottom margin. This will make the
% effective text area larger.
%
% IF YOU WISH TO REMOVE THE ``of LASTPAGE'' next to each page number,
% see the note about the +LP and -LP lines below. Comment out the +LP
% and uncomment the -LP.
%
% IF YOU WISH TO REMOVE PAGE NUMBERS, be sure that the includefoot line
% is uncommented and ALSO uncomment the \pagestyle{empty} a few lines
% below.
%

%% Use these lines for letter-sized paper
\usepackage[paper=letterpaper,
            includefoot, % Uncomment to put page number above margin
            marginparwidth=1in,     % Length of section titles
            marginparsep=.08in,       % Space between titles and text
            margin=0.5in,            % 0.75 inch margins
            includemp]{geometry}

%% Use these lines for A4-sized paper
%\usepackage[paper=a4paper,
%            %includefoot, % Uncomment to put page number above margin
%            marginparwidth=30.5mm,    % Length of section titles
%            marginparsep=1.5mm,       % Space between titles and text
%            margin=25mm,              % 25mm margins
%            includemp]{geometry}

%% More layout: Get rid of indenting throughout entire document
\setlength{\parindent}{0in}

%% This gives us fun enumeration environments. compactitem will be nice.
\usepackage{paralist}

%% Reference the last page in the page number
%
% NOTE: comment the +LP line and uncomment the -LP line to have page
%       numbers without the ``of ##'' last page reference)
%
% NOTE: uncomment the \pagestyle{empty} line to get rid of all page
%       numbers (make sure includefoot is commented out above)
%
\usepackage{fancyhdr,lastpage}
\pagestyle{fancy}
\pagestyle{empty}      % Uncomment this to get rid of page numbers
\fancyhf{}\renewcommand{\headrulewidth}{0pt}
\fancyfootoffset{\marginparsep+\marginparwidth}
\newlength{\footpageshift}
\setlength{\footpageshift}
          {0.5\textwidth+0.5\marginparsep+0.5\marginparwidth-2in}
\lfoot{\hspace{\footpageshift}%
       \parbox{4in}{\, \hfill %
                    \arabic{page} of \protect\pageref*{LastPage} % +LP
%                    \arabic{page}                               % -LP
                    \hfill \,}}

% Finally, give us PDF bookmarks
\usepackage{color,hyperref}
\definecolor{darkblue}{rgb}{0.0,0.0,0.4}
\hypersetup{colorlinks,breaklinks,
            linkcolor=darkblue,urlcolor=darkblue,
            anchorcolor=darkblue,citecolor=darkblue}

%%%%%%%%%%%%%%%%%%%%%%%% End Document Setup %%%%%%%%%%%%%%%%%%%%%%%%%%%%


%%%%%%%%%%%%%%%%%%%%%%%%%%% Helper Commands %%%%%%%%%%%%%%%%%%%%%%%%%%%%

% The title (name) with a horizontal rule under it
%
% Usage: \makeheading{name}
%
% Place at top of document. It should be the first thing.
\newcommand{\makeheading}[1]%
        {\hspace*{-\marginparsep minus \marginparwidth}%
         \begin{minipage}[t]{\textwidth+\marginparwidth+\marginparsep}%
                {\large \bfseries #1}\\[-0.15\baselineskip]%
                 \rule{\columnwidth}{1.5pt}%
         \end{minipage}}

% The section headings
%
% Usage: \section{section name}
%
% Follow this section IMMEDIATELY with the first line of the section
% text. Do not put whitespace in between. That is, do this:
%
%       \section{My Information}
%       Here is my information.
%
% and NOT this:
%
%       \section{My Information}
%
%       Here is my information.
%
% Otherwise the top of the section header will not line up with the top
% of the section. Of course, using a single comment character (%) on
% empty lines allows for the function of the first example with the
% readability of the second example.
\renewcommand{\section}[2]%
        {\pagebreak[2]\vspace{1\baselineskip}%
         \phantomsection\addcontentsline{toc}{section}{#1}%
         \hspace{0in}%
         \marginpar{
         \raggedright \scshape #1}#2}

% An itemize-style list with lots of space between items
\newenvironment{outerlist}[1][\enskip\textbullet]%
        {\begin{itemize}[#1]}{\end{itemize}%
         \vspace{-0.6\baselineskip}}

% An environment IDENTICAL to outerlist that has better pre-list spacing
% when used as the first thing in a \section
\newenvironment{lonelist}[1][\enskip\textbullet]%
        {\vspace{-\baselineskip}\begin{list}{#1}{%
        \setlength{\partopsep}{0pt}%
        \setlength{\topsep}{0pt}}}
        {\end{list}\vspace{-.6\baselineskip}}

% An itemize-style list with little space between items
\newenvironment{innerlist}[1][\enskip\textbullet]%
        {\begin{compactitem}[#1]}{\end{compactitem}}

% An environment IDENTICAL to innerlist that has better pre-list spacing
% when used as the first thing in a \section
\newenvironment{loneinnerlist}[1][\enskip\textbullet]%
        {\vspace{-\baselineskip}\begin{compactitem}[#1]}
        {\end{compactitem}\vspace{-.6\baselineskip}}

% To add some paragraph space between lines.
% This also tells LaTeX to preferably break a page on one of these gaps
% if there is a needed pagebreak nearby.
\newcommand{\blankline}{\vspace*{0pt}}

% Uses hyperref to link DOI
\newcommand\doilink[1]{\href{http://dx.doi.org/#1}{#1}}
\newcommand\doi[1]{doi:\doilink{#1}}


%%%%%%%%%%%%%%%%%%%%%%%% End Helper Commands %%%%%%%%%%%%%%%%%%%%%%%%%%%

%%%%%%%%%%%%%%%%%%%%%%%%% Begin CV Document %%%%%%%%%%%%%%%%%%%%%%%%%%%%

\usepackage{graphicx}
\begin{document}


\makeheading{Karan Jitendra Thakkar \hfill \href{http://in.linkedin.com/in/karanjthakkar}{Linkedin}}

\section{Contact}
%
% NOTE: Mind where the & separators and \\ breaks are in the following
%       table.
%
% ALSO: \rcollength is the width of the right column of the table
%       (adjust it to your liking; default is 1.85in).
%
\newlength{\rcollength}\setlength{\rcollength}{1.65in}%
\vspace{0.1cm}
\begin{tabular}[t]{@{}c|c|c}
\textit{Email:} \href{mailto:karanjthakkar@gmail.com}{karanjthakkar@gmail.com} & \textit{Website:} \href{karanjthakkar.co.nr}{karanjthakkar.co.nr} & \textit{Mobile:} $+$91-9637799260 \\

\end{tabular}

\hspace{-2.85cm}
\rule{1.17\linewidth}{1pt}%
\blankline

\section{Education}
\href{http://kitcoek.org/}{\textbf{KIT's College of Engineering}},
Kolhapur, India
\begin{innerlist}
\item[] B.E., \href{http://kitcoek.org/home.html} {Electronics and Telecommunication Engineering}
\hfill \textbf{August '08 - July '12} 
\item[] \textit{First Class with Distinction}, Average grade 70.68\%
\end{innerlist}

\blankline

\section{Online Courses}
\href{http://ai-class.com/}{\textbf{Introduction to Artificial Intelligence}} \small \hfill \textbf{October '12 - December '12}
\begin{innerlist}
\item[] Grade 79.30\% \hfill \small \href{http://karanjthakkar.files.wordpress.com/2012/12/statement-of-accomplishment.pdf}{\emph{Statement of Accomplishment}}
\end{innerlist}

\href{http://ai-class.com/}{\textbf{Computing for Data Analysis}} \small \hfill \textbf{September '13 - October '13}
\begin{innerlist}
\item[] Grade 89.00\% \hfill \small \href{http://karanjthakkar.files.wordpress.com/2013/11/computing-for-data-analysis.pdf}{\emph{Statement of Accomplishment}}
\end{innerlist}

\blankline

\section{Tech Skills}
\textbf{Languages:} C, C$+$$+$, C$\#$, Core Java, Python, Pro*C/C++, SQL, HTML5, CSS3, Javascript, Embedded C (8051/ARM)

\vspace{8pt}

\textbf{Tools:} \LaTeX, KiCAD, Keil $\mu$vision, Cadsoft EAGLE, Flash Magic, Visual Studio, Eclipse, git (\href{http://github.com/karanjthakkar}{github})

\vspace{8pt}

\textbf{Frameworks:} Sencha Touch, EnyoJS

\vspace{8pt}

\textbf{Libraries:} jQuery, Kinect SDK, OpenCV

\vspace{8pt}

\textbf{Hardware Platforms:} 8051, ARM, Arduino

\blankline

\section{Professional Experience}
\small \textbf{\href{http://www.tcs.com}{Tata Consultancy Services Ltd.}}, Pune, India \normalsize
\begin{innerlist}
\item[] \textbf{Assistant Systems Engineer \small \hfill January '13 - Present}
\item[] Working as a Hybrid Mobile Applications Developer for TCS Mobility Solutions Ltd.
\end{innerlist}

\vspace{8pt}

\small \textbf{\href{http://www.asep-championship.com}{Gade Autonomous Systems Pvt. Ltd.}}, Mumbai, India \normalsize
\begin{innerlist}
\item[] \textbf{Research Intern \small \hfill September '12 - November '12}
\item[] Developed a WPF application for a Gesture Based Picture Viewer using Microsoft Kinect
\end{innerlist}

\vspace{8pt}

\small \textbf{\href{http://www.callbackers.com/}{Callbackers Pvt. Ltd.}}, Chandigarh, India \normalsize
\begin{innerlist}
\item[] \textbf{Intern \small \hfill July '12 - August '12}
\item[] Designed ready-to-manufacture GSM and Xbee shields for Arduino Uno using EAGLE PCB design tool
\end{innerlist}

\blankline

\section{Projects}
\small \textbf{V'Me: A Gesture Based Picture Viewer application \hfill September '12 - November '12} \normalsize
\begin{innerlist}
\item[] Developed a WPF application for a Gesture Based Picture Viewer using Microsoft Kinect. Implemented functionalities such as: Swipe Left, Swipe Right, Zoom In, Zoom Out, Panning, Moving mouse cursor and Switching to different folders.\hfill \small \href{http://github.com/karanjthakkar/Gesture-Based-Picture-Viewer}{\textit{Code}} $|$ \href{http://youtu.be/v8SumS-I1qo}{\textit{Demo}} \normalsize
\end{innerlist}

\vspace{8pt}

\small \textbf{Jerry: An Intelligent Maze-Solving Robot \hfill August '11 - March '12} \normalsize
\begin{innerlist}
\item[] Worked on the design and construction of an efficient, intelligent, small, autonomous robot that can solve any arbitrary maze by finding the smallest and/or fastest path to the destination. This was done as a part of academic requirement for the final year. \small \hfill \href{http://karanjthakkar.files.wordpress.com/2012/03/jerry.pdf}{\textit{Report}} \normalsize
\end{innerlist}

\vspace{8pt}

\small \textbf{Hand Gesture Controlled Robot \hfill December '12} \normalsize
\begin{innerlist}
\item[] Developed a vehicle which was controlled by the motion of hand using an accelerometer
\end{innerlist}

\vspace{8pt}

\small \textbf{AC Mains Line Frequency Monitoring \hfill January '11 - March '11} \normalsize
\begin{innerlist}
\item[] Worked on design and construction of an electronic circuit for measuring the frequency of AC Mains power supply using PLL and take corrective measures if necessary.
\end{innerlist}

\vspace{8pt}

Other independent projects hosted on \href{http://resume.github.com/?karanjthakkar}{\textit{\textbf{Github} } }

\blankline

\section{Other Activities}
\vspace{-12pt}
\begin{innerlist} \itemsep 0pt
\item \textit{One of four} participants to represent college at \textit{Zonal Chess Competitions} '11
\item Placed 2$^{nd}$ in Inter-College Astronomy quiz, \textit{Krutika '11}
\item \emph{Technical Coordinator}, Association of Electronics and Telecommunication Students at KIT's College of Engineering, Kolhapur
\end{innerlist}


\end{document}
%%%%%%%%%%%%%%%%%%%%%%%%%% End CV Document %%%%%%%%%%%%%%%%%%%%%%%%%%%%%

